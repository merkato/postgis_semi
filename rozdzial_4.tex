\chapter{Wersjonowanie bazy danych}
	\section{Audit triggers}
Najprostsze rozwiązanie do nadzoru nad zmianami w bazie danych to tzw. audit triggers. Opiera się ono na tzw. wyzwalaczach (triggers), oraz funkcji realizującej operacje logowania zmian do tabeli. Dla operacji INSERT, UPDATE i DELETE tworzy się osobną procedurę logowania, a następnie wskazuje wyzwalaczowi kiedy ma zostać wywołana.
\begin{lstlisting}
CREATE TRIGGER tablename_audit
AFTER INSERT OR UPDATE OR DELETE ON tablename
FOR EACH ROW EXECUTE PROCEDURE audit.if_modified_func();
\end{lstlisting}	

Bardziej rozbudowana wersja tego rozwiązania znajduje się pod adresem 
\begin{lstlisting}
https://github.com/2ndQuadrant/audit-trigger
\end{lstlisting}
Jego instalacja jest bardzo prosta, wywołujemy skrypt SQL przy pomocy psql, a następnie wywołujemy następującym poleceniem dodanie tabeli do audytu.
\begin{lstlisting}
SELECT audit.audit_table('schema.tabela');
\end{lstlisting}
	Istnieje również wtyczka do QGIS pozwalająca na zarządzanie operacjami rollback względem zmian dokonywanych w tabeli. Nazywa się Postgres 91 plus Auditor.
	\section{Inne rozwiązania na rynku}
	Innym rozwiązaniem dostępnym na rynku jest między innymi Postgis Versioning opracowane przez firmę Oslandia. Jest to wtyczka dla QGISa, przy pomocy której wszystkie operacje związane z wersjonowaniem realizujemy w interfejsie graficznym. Wymaga ona pracy na grupach warstw. 
	Kolejnym rozwiązaniem jest wersjonowanie dostępne w QGIS 2.16+, do którego możemy się dostać z poziomu DB Managera
\chapter{Zapytania SQL - case studies}

\section{Zapytania Intersects}

Pierwsze zapytanie ma za zadanie porównanie dla obiektów punktowych numeru TERYT TERC zapisanego w zbiorze, z numerem TERYT gminy na której terenie znajduje się geometria obiektu.

\begin{lstlisting}
SELECT row_number() over (),
e.geometry,
e.teryt_gmin,
e.nazwa,
p.jpt_kod_je AS computed_teryt,
p.jpt_nazwa_ AS computed_nazwa 
FROM public.jednostki e, gminy p
WHERE ST_Intersects(e.geometry,p.geom) 
AND e.teryt_gmin!=left(p.jpt_kod_je,6);
\end{lstlisting}
Operację tą wykonujemy przy pomocy dwóch warunków, jednego przestrzennego, przecinania się obiektów (znajdowania się w takiej konfiguracji przestrzennej, w której mają jakąkolwiek część wspólną), oraz porównania ciągów znakowych przy pomocy operatora negacji, przy czym jeden z ciągów znakowych jest wstępnie przygotowany funkcją left("string",liczba)

Kolejne zadanie jest typową operacją \textbf{Zlicz punkty w poligonie}

\begin{lstlisting}
	SELECT COUNT(*) AS liczba, e.geom 
	FROM public.jednostki j, public.gminy e 
	WHERE st_intersects(j.geometry, e.geom) GROUP BY e.geom
\end{lstlisting}
Oprócz funkcji ST\_Intersects(geomA,geomB) używana jest również klauzula GROUP BY. Służy ona do agregowania tych wierszy danych które mają wspólne dane. W naszym przypadku częścią wspólną jest e.geom, zaś COUNT(*) ma za zadanie zliczenie każdorazowo ilości wierszy zaagregowanych. Warto zaznaczyć że w klauzuli GROUP BY muszą znaleźć się wszystkie obiekty (kolumny) wybrane w zapytaniu SELECT za wyjątkiem tych które są funkcjami agregującymi (COUNT, SUM, etc.)

\section{Tworzenie geometrii}
PostGIS pozwala na tworzenie nowych geometrii w oparciu o istniejące atrybuty. Szczególnymi funkcjami są ST\_Union(geom) i ST\_Collect(geom). Pierwsza wykonuje operację nazywaną \textit{Dissolve}, zaś druga tworzy tzw. GEOMETRYCOLLECTION. Zobaczmy na przykładzie połączenie wszystkich pól siatki jednokilometrowej z zbioru GUS "rozmieszczenie ludności", w których wartość jest niezerowa. 
\begin{lstlisting}
CREATE TABLE zamieszkane AS SELECT ST_Union(geom) AS geom FROM public.ludnosc_km WHERE tot > '0'
\end{lstlisting}

\section{Operatory zasięgu przestrzennego}

\begin{lstlisting}
&&
\end{lstlisting}
Operator ten zwraca stan logiczny dla porównania zasięgów przestrzennych tzw. bounding box. Najpopularniejszym  zastosowaniem tego operatora jest wstępne ograniczenie liczby rekordów i geometrii które będą później analizowane np przy pomocy funkcji ST\_Intersects()
\begin{lstlisting}
SELECT ST_Intersects(a.geom, (SELECT geom FROM b)) WHERE a.geom && b.geom;
\end{lstlisting}

\begin{lstlisting}
<->
\end{lstlisting}
Operator ten zwraca odległość obiektów, może służyć np. do sortowania wyników po odległości.
\begin{lstlisting}
SELECT row_number () OVER () as qgisid,* 
FROM warsaw.osm_emergency e 
WHERE type='fire_hydrant' 
AND st_intersects(e.geometry, (SELECT geometry FROM warsaw.osm_admin WHERE id='14'))
ORDER BY e.geometry <-> (SELECT geometry FROM warsaw.osm_emergency WHERE name='Nazwa OSP' LIMIT 1)	
\end{lstlisting}
Jeśli potrzebujemy uzyskać samą wartość odległości, powinniśmy wykorzystać funkcję ST\_Distance(g1,g2). 

\section{Zarządzanie danymi}
Pierwsze zadanie jakie omówimy to łączenie dwóch tabel do jednego widoku. Wykorzystujemy w tym celu klauzulę UNION. Aby połączyć więcej niż dwie tabele, każdy kolejny obiekt powinien być zagnieżdżony w kolejnym nawiasie.

\begin{lstlisting}
CREATE VIEW szkolenie.rule_them_all AS (
(SELECT * FROM szkolenie.kielecki 
UNION 
SELECT * FROM szkolenie2.skarzyski)
UNION 
SELECT * FROM szkolenie.buski)
UNION
SELECT * FROM szkolenie2.konecki
\end{lstlisting}